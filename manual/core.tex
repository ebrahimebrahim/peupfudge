Peupfudge is a generic framework for tabletop role-playing games.
This document contains the core rules, which can later be modified or extended by modules.
The game master (GM) constructs or finds a world that can use the Peupfudge framework,
and then runs campaigns within that world. 


Players will face challenges, and they will have to rely on the capabilities of their characters to confront them.
Peupfudge is a system that crudely quantifies these challenges and capabilities in order to compare them and weave an interesting narrative.

The main die type used here is the fudge die, which can yield $-1$, $0$, or $+1$ with equal chances.
We denote a fudge die roll by ``dF.'' The result of a roll of $N$ fudge dice ($N$dF) is the sum of the results of the individual die rolls.

\section*{Characters}
A character is an actor in the world.
Some characters are controlled by players (PCs) and some characters are controlled by the GM (NPCs).
Characters are represented by their traits, which we list below.
PC traits are managed by players on their character sheets, while NPC traits are managed by the GM.
When creating a world or campaign, the GM determines the traits that will appear on character sheets.

There are five types of traits:
\begin{itemize}
\item
Characteristics describe identity and background that mainly impact the narrative aspects of a campaign.
Examples: name, race, species, tribe, height, gender, favorite food, appearance, backstory.
\item
Abilities represent improvable traits that play a role in action resolution.
Examples: strength, intelligence, climbing, legal knowledge, cartography, neurosurgery.
We will soon describe how abilities work.
\item
Inventory is the collection of items on a character’s person.
Examples: helmet, potion, gold coins.
More detail can be included as needed; for example, one could write “helmet (equipped)” or “helmet (damaged).”
\end{itemize}
If a trait doesn’t fit into the three categories above, then it belongs in one of the following two:
\begin{itemize}
\item
Statuses are otherwise unclassified traits that need constant tracking.
Examples: health, mana, hunger, reputation.
When introducing a status, the GM should decide what states it can take, what causes the state to change, and what effect each state has.
\item
Properties are otherwise unclassified traits that only need to be tracked when they apply to a character.
Examples: deaf, blessed, stunned, short-tempered, one-armed.
\end{itemize}

For tips on setting up the traits, refer to the campaign setup checklist in the appendix.

\subsection*{Abilities}
Before a campaign begins, the GM prepares a set of abilities that are tracked for each character.
Each ability is associated with an integer level that represents how good a character is at the ability.
The ability level is a factor in the character’s chance of success when attempting to perform an action dependent on the ability.
Typically, the initial level for an “untrained” ability would be 0, and there is no hard upper limit for the level as it increases.

The level of an ability can be raised by allocating experience points (XP) to the ability.
The XP cost of increasing the level of an ability is 2 to the power of the current level.
The GM may decide how and when to award XP to characters, but it makes sense to place XP rewards after the completion of narrative ``chunks.''
As soon as XP is awarded, it should be distributed by each player among their abilities.
The GM may want to restrict XP allocation to the subset of abilities that each player actually used in the completion of the narrative chunk.

Players may spend less XP on an ability than it would cost to level it up.
In this case the allocated XP is recorded and the ability is only leveled up once it has accumulated enough XP.
Allocated XP is considered to be spent and cannot later be transferred to a different ability.

XP is to be interpreted as the product of practice time and practice quality, which we will refer to as simply practice.
Each unit of XP corresponds to a certain amount of practice.
The exact amount of practice contained in each unit of XP is decided implicitly the first time that the GM awards XP,
and it crystallizes as the GM continues to award XP in a consistent pattern.

Abilities start at level 0 during character creation.
The GM provides starting XP to each character based on the practice the character may have gathered throughout their life before the start of the campaign.
If a character has a natural talent in a given ability,
then they could get some ability levels for free prior to XP allocation; this would represent a sort of head start in the ability.

Ability level is a function of XP, and XP represents practice.
But ability level is ultimately meant to represent a character's proficiency.
Of course practice generally improves proficiency, but characters with the same amount of practice can end up with different levels of proficiency.
For example, a hobbit and an ogre may get the same amount of strength practice, but the hobbit will still be weaker than the ogre.
To incorporate this into Peupfudge, a property ``ogre strength'' may be added to the ogre’s character sheet.
This property would add a certain number to the strength level, but it would do so in a way that should not affect the XP cost for leveling up.
To keep track of this, the ogre’s strength level can be written in the character sheet as a sum: [unmodified level] + [modifier].
When the ogre needs to use their strength, they use the sum.
When the ogre wants to spend XP and level up their strength, they use the unmodified level to determine the cost.

Equivalently, the hobbit could instead take on a “hobbit strength” property that reduces the strength level in a similar way.
Whichever being ends up not taking on any modifier becomes the standard being for the purposes of level interpretation.
How strong is a character with level 5 strength?
It is as strong as a standard being that spent 5 levels worth of practice on its strength.

\begin{center}
\begin{tabular}{|l|l|l|l|l|l|l|l|l|l|l|l|}
\hline
Level    & 0 & 1 & 2 & 3 & 4  & 5  & 6  & 7   & 8   & 9   & $\ldots$ \\ \hline
XP Cost  & 0 & 1 & 2 & 4 & 8  & 16 & 32 & 64  & 128 & 256 & $\ldots$ \\ \hline
Total XP & 0 & 1 & 3 & 7 & 15 & 31 & 63 & 127 & 255 & 511 & $\ldots$ \\ \hline
\end{tabular}
\end{center}


\section*{Actions}
When a character attempts a task that has a possibility of failure, the outcome is governed by ability, difficulty, and a dice roll. 

We use the term aiding factors to refer to the factors leading to the possibilities of success or failure for an action,
but only those factors that are due to the character attempting the action.
For those factors that are independent of the character attempting the action we use the term difficulty factors.
The aiding factors for a character attempting to climb a cliff are things like the climbing skill of the character and the fact that they have a grappling hook.
The steepness of the cliff and the availability of footholds would be difficulty factors.
The difficulty factors for a character attempting to punch someone in a brawl might include the defensive abilities of the opponent,
while the punching abilities of the character would be aiding factors.
A rule of thumb is that if a factor can be removed from the picture by changing who is attempting an action, then it is an aiding factor.
Otherwise, it is a difficulty factor.

Actions are resolved by considering an attempt to be successful if

$$L + A + R \geq D,$$

where
\begin{itemize}
\item
$L$ is the level of an associated ability for the character attempting the task, if there is one.
The set of abilities chosen for the campaign should be tailored to the types of actions that players are expected to attempt.
If there isn’t a very suitable ability but there is a close enough one, then the level of the close ability can be used with a deduction. 
\item
$A$ is a modifier for any other aiding factors, like the grappling hook.
\item
$R$ is the result of a dice roll, $N$dF.
The dice roll represents the dependence of the outcome on factors that are not modeled in the game or that are otherwise unpredictable.
The default is $4$dF, and this can be modified if more or less noise seems appropriate.
\item
$D$ is the difficulty level of the task, a summary of the difficulty factors.
Some judgement is needed on the part of the GM in the determination of $D$.
A difficulty level of $D$ corresponds to a task that someone with an ability level of $D$ would be expected to successfully execute about half the time.
\end{itemize}

The value of $R$ can feed directly into the narrative; it can be fun to assign extreme narrative interpretations to extreme dice rolls.
The same goes for $L + A + R - D$, which represents actual performance given all the information.

When setting $D$, the GM should avoid the pitfall of assessing a task only by comparison to similar tasks.
For example, a neurosurgery task should not be given a lower $D$ just because it is easy “for a neurosurgery.” Note that $D$ ultimately gets compared to an ability level L, which is a function of experience.
Since neurosurgery itself is hard (in the sense of requiring a lot of experience), the $D$ associated with many neurosurgery tasks, even the relatively easy ones, should be high.
It might be a good idea when setting up the abilities for a given campaign to determine what level for each skill is considered “poor,” “decent,” “excellent,” etc.

If the difficulty factors for an action are due to some other character's abilities, for example during a race between multiple characters, then the action is opposed.
There are a couple of modifications one could make for opposed actions:
For fun, the physical dice rolling can be done by all participants whose characters are involved, with players rolling for their characters and the GM rolling for NPCs.
Instead of comparing one character’s $L + A + R$ with a $D$ based on the difficulty of the task, one can compare $L + A + R$ for each character to determine their performance relative to one another.
When the $L + A + R$ of multiple characters is equal, this should be interpreted as a tie rather than a success for any of the tied characters.

Happy Peupfudging!

