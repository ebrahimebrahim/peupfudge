\section{Examples}

\newcounter{example}[section]
\newenvironment{example}{\refstepcounter{example}\paragraph{Example \theexample:}}{}

\begin{example}
You are a mage who is near the end of an arduous journey through a dungeon, looking for a golden teapot at the end of it.
You open the door to the final room and enter into a long, narrow room largely taken up by a gaping abyss.
On the other side of the abyss you see the teapot!
You fish through your pack for your Scroll of Magical Bridge.
Your level in the Construction Magic ability is $6$,
but the GM determines
$D = 7$
by considering that a standard being with $7$ levels worth of Construction Magic XP would succeed half the time at casting this spell over an abyss of that size.
Additionally, you recently fell ill after eating the wrong kind of dungeon mushroom,
and this gave a property ``Shroom Belly'' that makes it harder to focus on magic,
granting $A = -1$ for magical actions.
You roll $4$dF... and you get $+3$!
$L = 6$, $A = -1$, $R = +3$, $D = 7$, and $8 \geq 7$.
You successfully create the bridge, walk across it, and grab the golden teapot!
The GM awards you $10$ XP for completion of this narrative chunk.
You decide to allocate $7$ XP towards Construction Magic, and $3$ XP towards Mycology (another ability you used in this dungeon, albeit not successfully).
Construction Magic already had $22$ XP allocated to it (out of the $64$ XP required to raise it to level $7$), and now has $29$.
Mycology was at level $0$, and has now been raised to level $2$.
\end{example}

\begin{example}
You are the GM and the players have just fought their way into a dungeon room with a closed door at the opposite end.
The door is cracked open from the other side, just barely revealing a stack of nasty gremlins.
Adam announces that his archer character attempts to fire an arrow into the narrow crack.
As the GM, you need to set the difficulty $D$ for this action.
You decide that in order to have about a $50\%$ chance of success, one would need about a year of dedicated archery training.
Earlier in the campaign a different character did a month of dedicated training and received $2$ XP for her archery ability.
A year of dedicated archery training then corresponds to $24$ XP, which corresponds to a level of about $4$.
So you let Adam know that $D=4$ for this action.
``That’s pretty harsh man,'' Adam complains.
``It’s not that hard for an archery task.''
Being the wise GM that you are, you understand that $D$ comes down to the amount of needed practice and not a comparison with similar tasks.
Once you explain how you determined $D$, Adam has a further complaint:
``My archery ability level is $2$, which corresponds to $3$ XP.
By your convention that’s $1.5$ months of archery practice… but my character has been an archer for like $3$ months!
What gives?!
Why do I get less XP?!'' After calming Adam down, you explain that his character’s practice wasn’t dedicated archery training.
His $3$ months of undedicated practice is being treated as though it were $1.5$ months of dedicated training.
Finally, Adam needs to roll to resolve the action.
With $L=2$ and $D=4$, it appears that Adam has to roll at least $+2$ to succeed.
Rolling $4$dF, this gives him about a $1/5$ chance of success.
Adam rolls his $4$dF and gets $R=-1$, failing the shot.
Since $-1$ is not such a surprising roll, you do not interpret Adam’s failure in a dramatically unlucky way.
His relative performance, on the other hand, is quite low: $L+A+R-D = 2+0-1-4 = -3$.
And so you narrate that the arrow appears to have had little hope of hitting its target,
and the gremlins all snicker at Adam’s character.
\end{example}
