\section{Probability Reference}

The following table shows the probabilities of success when rolling different amounts of fudge dice.
Each entry is the probability that $N$dF will be $\geq D$, for some $N$ and $D$.

\begin{center}
\input{ndf_table.tex}
\end{center}

The values in the highlighted column are close to $0.5$,
hence the rule of thumb:
A task of difficulty level $D$ is one that someone with an ability level of $D$ would be expected to successfully execute about half the time.

The highlighted row corresponds to $4$dF, a nice standard number of dF to roll.
By rolling more or fewer dF, one can vary the spread of the distribution of outcomes.
The outcome of rolling $N$dF approaches a normal distribution as $N$ increases,
with the standard deviation being a constant multiple of $\sqrt{N}$.
Note that the spread of the distribution does not vary linearly with the number of dice.
There’s a bigger difference between rolling $2$dF and rolling $4$dF than there is between rolling $4$dF and rolling $6$dF.
\begin{center}
\includegraphics[scale=0.7]{ndf_plot.pdf}
\end{center}
